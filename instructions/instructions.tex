%
% Complete documentation on the extended LaTeX markup used for Insight
% documentation is available in ``Documenting Insight'', which is part
% of the standard documentation for Insight.  It may be found online
% at:
%
%     http://www.itk.org/

\documentclass{InsightArticle}

\usepackage[dvips]{graphicx}
%\usepackage{listing}						
\usepackage{listings}	
\usepackage{wrapfig}
\usepackage{amssymb,amsmath}
\usepackage{multirow,booktabs,array}
\usepackage{listings}
\usepackage{color}
				
%%%%%%%%%%%%%%%%%%%%%%%%%%%%%%%%%%%%%%%%%%%%%%%%%%%%%%%%%%%%%%%%%%
%
%  hyperref should be the last package to be loaded.
%
%%%%%%%%%%%%%%%%%%%%%%%%%%%%%%%%%%%%%%%%%%%%%%%%%%%%%%%%%%%%%%%%%%
\usepackage[
bookmarks,
bookmarksopen,
backref,
colorlinks,linkcolor={blue},citecolor={blue},urlcolor={blue},
]{hyperref}




\graphicspath{{./figures/}}


 
\definecolor{dkgreen}{rgb}{0,0.6,0}
\definecolor{gray}{rgb}{0.5,0.5,0.5}
\definecolor{mauve}{rgb}{0.58,0,0.82}

\lstset{ %
  language=bash,                % the language of the code
  basicstyle=\footnotesize,           % the size of the fonts that are used for the code
  numbers=left,                   % where to put the line-numbers
  numberstyle=\tiny\color{gray},  % the style that is used for the line-numbers
  stepnumber=2,                   % the step between two line-numbers. If it's 1, each line 
                                  % will be numbered
  numbersep=5pt,                  % how far the line-numbers are from the code
  backgroundcolor=\color{white},      % choose the background color. You must add \usepackage{color}
  showspaces=false,               % show spaces adding particular underscores
  showstringspaces=false,         % underline spaces within strings
  showtabs=false,                 % show tabs within strings adding particular underscores
  frame=single,                   % adds a frame around the code
  rulecolor=\color{black},        % if not set, the frame-color may be changed on line-breaks within not-black text (e.g. commens (green here))
  tabsize=2,                      % sets default tabsize to 2 spaces
  captionpos=b,                   % sets the caption-position to bottom
  breaklines=true,                % sets automatic line breaking
  breakatwhitespace=true,        % sets if automatic breaks should only happen at whitespace
  prebreak=\textbackslash, breakindent=7pt,
  title=\lstname,                   % show the filename of files included with \lstinputlisting;
                                  % also try caption instead of title
  keywordstyle=\color{blue},          % keyword style
  commentstyle=\color{dkgreen},       % comment style
  stringstyle=\color{mauve},         % string literal style
  escapeinside={\%*}{*)},            % if you want to add a comment within your code
  morekeywords={*,...}               % if you want to add more keywords to the set
}

\lstdefinestyle{bash}{language=bash} 
\lstdefinestyle{perl}{language=perl}
\lstdefinestyle{bibtex}{language=bibtex}
 

%  This is a template for Papers to the Insight Journal. 
%  It is comparable to a technical report format.

% The title should be descriptive enough for people to be able to find
% the relevant document. 
\title{PICSL lab bibliography style and technical instructions}

% 
% NOTE: This is the last number of the "handle" URL that 
% The Insight Journal assigns to your paper as part of the
% submission process. Please replace the number "1338" with
% the actual handle number that you get assigned.
%
%\newcommand{\IJhandlerIDnumber}{}

% Increment the release number whenever significant changes are made.
% The author and/or editor can define 'significant' however they like.
\release{0.1}

% At minimum, give your name and an email address.  You can include a
% snail-mail address if you like.
\author{Philip Cook, Ben Kandel, John Wu}
\authoraddress{Penn Image Computing And Science Laboratory\\
University of Pennsylvania}

\begin{document}

%
% Add hyperlink to the web location and license of the paper.
% The argument of this command is the handler identifier given
% by the Insight Journal to this paper.
% 
%\IJhandlefooter{\IJhandlerIDnumber}


\ifpdf
\else
   %
   % Commands for including Graphics when using latex
   % 
   \DeclareGraphicsExtensions{.eps,.jpg,.gif,.tiff,.bmp,.png}
   \DeclareGraphicsRule{.jpg}{eps}{.jpg.bb}{`convert #1 eps:-}
   \DeclareGraphicsRule{.gif}{eps}{.gif.bb}{`convert #1 eps:-}
   \DeclareGraphicsRule{.tiff}{eps}{.tiff.bb}{`convert #1 eps:-}
   \DeclareGraphicsRule{.bmp}{eps}{.bmp.bb}{`convert #1 eps:-}
   \DeclareGraphicsRule{.png}{eps}{.png.bb}{`convert #1 eps:-}
\fi


\maketitle


\ifhtml
\chapter*{Front Matter\label{front}}
\fi



% The abstract should be a paragraph or two long, and describe the
% scope of the document.
\begin{abstract}
\noindent This document explains how to update the lab master bibliography and website.
\end{abstract}

\tableofcontents
\newpage

\section*{Introduction}

We maintain a lab-wide Bibtex file for multiple uses, including updating the website but also for grant applications and departmental reporting requirements. Each member of the lab needs to contribute their own publications to the bibliography, and we must all use a consistent system to avoid divergence in citation style.

\subsection{Setup}

To get started, perform the following one-time setup steps.

\begin{enumerate}
\item Register for a free GitHub account, and email your username to Phil
\item Install Jabref 2.9.2 for your platform (see section on Jabref below)
\end{enumerate}


\section{The Git repository}

The picslbib Git repository contains the Bibtex and this document. You may clone via https or ssh, depending on how your Git is configured. To clone via https, do

\begin{lstlisting}[style=bash]

  $ git clone https://github.com/cookpa/picslbib.git

  

\end{lstlisting}



\bibliographystyle{plain}
\bibliography{../picsl.bib}


\end{document}

