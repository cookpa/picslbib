%
% Complete documentation on the extended LaTeX markup used for Insight
% documentation is available in ``Documenting Insight'', which is part
% of the standard documentation for Insight.  It may be found online
% at:
%
%     http://www.itk.org/

\documentclass{InsightArticle}

\usepackage[dvips]{graphicx}
%\usepackage{listing}						
\usepackage{listings}	
\usepackage{wrapfig}
\usepackage{amssymb,amsmath}
\usepackage{multirow,booktabs,array}
\usepackage{listings}
\usepackage{color}
\usepackage[T1]{fontenc}
				
%%%%%%%%%%%%%%%%%%%%%%%%%%%%%%%%%%%%%%%%%%%%%%%%%%%%%%%%%%%%%%%%%%
%
%  hyperref should be the last package to be loaded.
%
%%%%%%%%%%%%%%%%%%%%%%%%%%%%%%%%%%%%%%%%%%%%%%%%%%%%%%%%%%%%%%%%%%
\usepackage[
bookmarks,
bookmarksopen,
backref,
colorlinks,linkcolor={blue},citecolor={blue},urlcolor={blue},
]{hyperref}




\graphicspath{{./figures/}}


 
\definecolor{dkgreen}{rgb}{0,0.6,0}
\definecolor{gray}{rgb}{0.5,0.5,0.5}
\definecolor{mauve}{rgb}{0.58,0,0.82}

\lstset{ %
  language=bash,                % the language of the code
  basicstyle=\footnotesize,           % the size of the fonts that are used for the code
  numbers=left,                   % where to put the line-numbers
  numberstyle=\tiny\color{gray},  % the style that is used for the line-numbers
  stepnumber=2,                   % the step between two line-numbers. If it's 1, each line 
                                  % will be numbered
  numbersep=5pt,                  % how far the line-numbers are from the code
  backgroundcolor=\color{white},      % choose the background color. You must add \usepackage{color}
  showspaces=false,               % show spaces adding particular underscores
  showstringspaces=false,         % underline spaces within strings
  showtabs=false,                 % show tabs within strings adding particular underscores
  frame=single,                   % adds a frame around the code
  rulecolor=\color{black},        % if not set, the frame-color may be changed on line-breaks within not-black text (e.g. commens (green here))
  tabsize=2,                      % sets default tabsize to 2 spaces
  captionpos=b,                   % sets the caption-position to bottom
  breaklines=true,                % sets automatic line breaking
  breakatwhitespace=true,        % sets if automatic breaks should only happen at whitespace
  prebreak=\textbackslash, breakindent=7pt,
  title=\lstname,                   % show the filename of files included with \lstinputlisting;
                                  % also try caption instead of title
  keywordstyle=\color{blue},          % keyword style
  commentstyle=\color{dkgreen},       % comment style
  stringstyle=\color{mauve},         % string literal style
  escapeinside={\%*}{*)},            % if you want to add a comment within your code
  morekeywords={*,...}               % if you want to add more keywords to the set
}

\lstdefinestyle{bash}{language=bash} 
\lstdefinestyle{perl}{language=perl}
\lstdefinestyle{bibtex}{language=bibtex}
 

%  This is a template for Papers to the Insight Journal. 
%  It is comparable to a technical report format.

% The title should be descriptive enough for people to be able to find
% the relevant document. 
\title{PICSL lab bibliography style instructions}

% 
% NOTE: This is the last number of the "handle" URL that 
% The Insight Journal assigns to your paper as part of the
% submission process. Please replace the number "1338" with
% the actual handle number that you get assigned.
%
%\newcommand{\IJhandlerIDnumber}{}

% Increment the release number whenever significant changes are made.
% The author and/or editor can define 'significant' however they like.
\release{0.2}

% At minimum, give your name and an email address.  You can include a
% snail-mail address if you like.
%\author{}
%\authoraddress{Penn Image Computing And Science Laboratory\\
%University of Pennsylvania}

\begin{document}

%
% Add hyperlink to the web location and license of the paper.
% The argument of this command is the handler identifier given
% by the Insight Journal to this paper.
% 
%\IJhandlefooter{\IJhandlerIDnumber}


\ifpdf
\else
   %
   % Commands for including Graphics when using latex
   % 
   \DeclareGraphicsExtensions{.eps,.jpg,.gif,.tiff,.bmp,.png}
   \DeclareGraphicsRule{.jpg}{eps}{.jpg.bb}{`convert #1 eps:-}
   \DeclareGraphicsRule{.gif}{eps}{.gif.bb}{`convert #1 eps:-}
   \DeclareGraphicsRule{.tiff}{eps}{.tiff.bb}{`convert #1 eps:-}
   \DeclareGraphicsRule{.bmp}{eps}{.bmp.bb}{`convert #1 eps:-}
   \DeclareGraphicsRule{.png}{eps}{.png.bb}{`convert #1 eps:-}
\fi


\maketitle


\ifhtml
\chapter*{Front Matter\label{front}}
\fi



% The abstract should be a paragraph or two long, and describe the
% scope of the document.
\begin{abstract}
\noindent This document contains style instructions for adding publications to the lab bibliography.
\end{abstract}

\tableofcontents
\newpage

\section{Introduction}

This style guide must be followed to ensure that the lab bibliography maintains a consistent look and feel. For technical instructions on how to configure Jabref and upload references, see \code{techguide.pdf}.


\section{Adding new publications}

Editing the bibliography consists mostly of adding new papers. For published papers, use Jabref's web search. Use the drop down box to change the search source. If the journal or conference is listed on Pubmed, search Medline. For IEEE journals not on Medline, search with IEEEXplore. Do not use Google Scholar because the Bibtex is automatically generated and can be inconsistent.

Each article only needs to be added once. If your search returns articles already in the database, these articles will be highlighted with a ``D". Do not import duplicates.


\section{Adding in press articles}

To add a publication that is accepted but not yet published, you will need to add an entry by hand. To do this, you should copy an existing entry from the same journal or conference. Edit the ``year" field to say ``In press", leave volume and pages blank. This is not valid Bibtex but it will render acceptably on the website. Of course this means you need to find and remove the entry once the paper is published. It might be better to just add in press articles manually on the website, then add the Bibtex once the paper is indexed in Medline. 


\section{Unlisted publications}

Many conferences are not listed in Medline or IEEEXplore.  We define our own standards and provide example Bibtex in the following sections. New conferences and journals can be added as necessary. 

If you are the first to add a conference, you can define the style. Place the template below so others can use it. 

\section{ISMRM}

This style follows the ISMRM's citation database in PubMed format. The ISMRM has made some Bibtex listings available, but they contain important errors including re-ordering of the authors! Use with caution. Here is the style template for importing your abstracts.

\begin{lstlisting}[language=TeX]
@INPROCEEDINGS{AuthorYearISMRM,
  author = {Author, A. and AnotherAuthor, B.},
  title = {Abstract title},
  booktitle = {Proceedings Nth Scientific Meeting, International Society for Magnetic
	Resonance in Medicine, City},
  year = {YYYY},
  pages = {Abstract number goes here},
  abstract = {The short abstract provided for the ISMRM abstract index},
  keywords = {CityAbstractNumber},
  timestamp = {YYYY.MM.DD}
}
\end{lstlisting}



\bibliographystyle{ieee}
\bibliography{../picsl.bib}


\end{document}

